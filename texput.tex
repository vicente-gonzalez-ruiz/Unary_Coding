%%% Local Variables:
%%% mode: latex
%%% TeX-master: "<none>"
%%% End:

\title{Unary coding}

\author{Vicente González Ruiz}

\maketitle

\section{Ideas}
\begin{itemize}
\item
  Used then the distribution of probability of the symbols follows a
  \href{https://en.wikipedia.org/wiki/Geometric_distribution}{geometric
  distribution}.
\item
  It is a particular case of the Huffman code where the number of bits
  of each code-word (minus one) is equal to the index of the symbol in
  the source alphabet. Example:
\end{itemize}

\img{150}{graphics/unary_coding_example.png}

\begin{itemize}
\item
  The unary coding is only optimal when (see Equation
  \(\text{Eq:symbol_information}\))

  \begin{equation}
    p(s) = 2^{-(s+1)} \tag{Eq:Unary}
  \end{equation}

  where \(s=0,1,\cdots\).
\end{itemize}

\img{800}{graphics/unary_coding.png}
